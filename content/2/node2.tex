En el nodo 2 se han implementado también todas las tareas y requisitos pedidos
en el documento de requisitos.

La tarea de actualización del modo de funcionamiento viene definida por el código
\ref{lst:modetask}:

\lstinputlisting[style=C,caption={Tarea esporádica que controla el modo.},label={lst:modetask},linerange={110-124},firstnumber=110]{source-code/node2/Src/main.c}

Dicha tarea espera a un semáforo binario que indica que se ha producido una interrupción
(línea 119). Una vez desbloqueada, incrementa el modo hasta un máximo de `2' (línea 120)
y finalmente actualiza el objeto protegido \texttt{mode} (códigos \ref{lst:mode_h} y
\ref{lst:mode_c}).

Por otra parte, como hace falta liberar el semáforo cuando se produce la interrupción
del pulsador, es necesario definir un nuevo fragmento del código que habilite a la
placa para esa tarea (código \ref{lst:exti}):

\lstinputlisting[style=C,caption={Rutina de tratamiento de interrupciones.},label={lst:exti},linerange={663-675},firstnumber=663]{source-code/node2/Src/main.c}

Cuando se activa el GPIO3 se ``devolverá'' el semáforo, lo que permitirá que la tarea
esporádica pueda adquirirlo y realizar su ejecución. Sin embargo, cuando lo intente
adquirir de nuevo se bloqueará y hasta que no se repita este proceso quedará a la
espera de que se libere el semáforo.
