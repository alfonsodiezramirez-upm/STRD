En el nodo 2 se han implementado también todas las tareas y requisitos pedidos
en el documento de requisitos.

La tarea de actualización del modo de funcionamiento viene definida por el código
\ref{lst:modetask}:

\lstinputlisting[style=C,caption={Tarea esporádica que controla el modo.},label={lst:modetask},linerange={110-124},firstnumber=110]{source-code/node2/Src/main.c}

Dicha tarea espera a un semáforo binario que indica que se ha producido una interrupción
(línea 119). Una vez desbloqueada, incrementa el modo hasta un máximo de `2' (línea 120)
y finalmente actualiza el objeto protegido \texttt{mode} (códigos \ref{lst:mode_h} y
\ref{lst:mode_c}).

Por otra parte, como hace falta liberar el semáforo cuando se produce la interrupción
del pulsador, es necesario definir un nuevo fragmento del código que habilite a la
placa para esa tarea (código \ref{lst:exti}):

\lstinputlisting[style=C,caption={Rutina de tratamiento de interrupciones.},label={lst:exti},linerange={663-675},firstnumber=663]{source-code/node2/Src/main.c}

Cuando se activa el GPIO3 se ``devolverá'' el semáforo, lo que permitirá que la tarea
esporádica pueda adquirirlo y realizar su ejecución. Sin embargo, cuando lo intente
adquirir de nuevo se bloqueará y hasta que no se repita este proceso quedará a la
espera de que se libere el semáforo.

Por otra parte, la tarea que se encarga de detectar si hay o no volantazos viene definida
por el código \ref{lst:wheel-task}:

\lstinputlisting[style=C,caption={Tarea periódica de control del giro del volante.},label={lst:wheel-task},linerange={126-159},firstnumber=126]{source-code/node2/Src/main.c}

En dicha tarea se lee desde el ADC (canal 0) y se actualiza el valor de la posición
del volante en la variable \texttt{actual} (líneas 139 -- 146); a continuación se
conserva el dato en el objeto protegido \texttt{symptoms} (línea 147) y se comprueba,
accediendo a la velocidad actual, si el conductor está dando volantazos o no. Esto se
realiza en las líneas 148 -- 155, en donde se aprovechan los recursos provistos por
el objeto protegido \texttt{wheel} para verificar si la se ha pegado un volantazo
(códigos \ref{lst:symptoms_h} y \ref{lst:symptoms_c}). Si durante 13 iteraciones no se ha
reseteado el contador (no ha habido volantazo) se quita el síntoma.

En lo referente a la tarea que identifica si el volante está siendo sujeto o no, se
define mediante el código \ref{lst:grabtask}:

\lstinputlisting[style=C,caption={Tarea periódica de control de sujección del volante.},label={lst:grabtask},linerange={161-174},firstnumber=161]{source-code/node2/Src/main.c}

Dicha tarea sencillamente leerá el valor del GPIO correspondiente (línea 170) y actualizará
el valor en el objeto protegido (línea 171), definido en los códigos \ref{lst:symptoms_h}
y \ref{lst:symptoms_c}.

Para trabajar con la inclinación de la cabeza, se ha de acceder a los valores provistos por
el giroscopio integrado en la placa. El código de la tarea viene definido por
\ref{lst:headtask}:

\lstinputlisting[style=C,caption={Tarea periódica de control de la inclinación de la cabeza.},label={lst:headtask},linerange={176-211},firstnumber=176]{source-code/node2/Src/main.c}

De todo el código anterior, los datos se actualizan en la línea 207 en donde guardan en
el objeto protegido \texttt{symptoms} (códigos \ref{lst:symptoms_h} y \ref{lst:symptoms_c}).

Finalmente, la tarea de riesgos. Esta tarea accede a todos los objetos protegidos y
además obtiene información mediante el CANBus, por lo que su programación es delicada.
Dicha tarea viene definida por el código \ref{lst:riskstask}:

\lstinputlisting[style=C,caption={Tarea periódica de control de riesgos.},label={lst:riskstask},linerange={213-288},firstnumber=213]{source-code/node2/Src/main.c}

Lo primero que se realiza en esta tarea es comprobar si el modo de funcionamiento de las
alarmas es menor a 2 (línea 241). Esto quiere decir que las alarmas están habilitadas al completo
o parcialmente (un valor de 2 indicaría que ninguna alarma está activa).

Si hay un modo de alarma activado, se empiezan a realizar las comprobaciones:
\begin{itemize}
  \item En la línea 242 se comprueba que existe una inclinación simultánea en los ejes $X$
        e $Y$ y que además no se tiene el volante sujeto. Si está el estado de alarma
        básico activo (línea 243), se enciende la luz amarilla y se emite un pitido con
        intensidad 1.
  \item En la línea 250 se comprueba si la cabeza está inclinada al menos en uno de los ejes,
        si el volante está sujeto y si se circula a más de $70 \nicefrac{km}{h}$, en cuyo
        caso se encendería la luz amarilla.
  \item Si se inclina la cabeza más de 30º en el eje $X$ y se están dando volantazos (línea
        257) entonces se emite un pitido nivel dos y se enciende la luz amarilla.
  \item Si no se ha dado ninguna situación de riesgo (\texttt{risk\_count == 0}), se
        apaga el pitido y la luz amarilla (líneas 265 -- 268). Si, por el contrario, se
        ha producido al menos dos riesgos se activa el pitido nivel dos y se enciende
        la luz roja (líneas 269 -- 275). En otro caso, se apagan.
\end{itemize}

Con esta lógica, el diagrama \ref{fig:risks-diagram} queda implementado a nivel de código.
En este caso, se utilizan los objetos protegidos de síntomas (códigos \ref{lst:symptoms_h}
y \ref{lst:symptoms_c}), los que contienen los datos recibidos por el CANBus (\texttt{node1}
definido por \ref{lst:node1_h} y \ref{lst:node1_c}) y el modo (códigos \ref{lst:mode_h} y
\ref{lst:mode_c}).

Finalmente, para poder recibir los datos por el CANBus se han creado dos tareas esporádicas
las cuales esperan un evento para poder actualizar los valores contenidos en los objetos
protegidos. Dichas funciones vienen definidas por el código \ref{lst:canbustasks}:

\lstinputlisting[style=C,caption={Tareas esporádicas de gestión del CANBus.},label={lst:canbustasks},linerange={290-324},firstnumber=290]{source-code/node2/Src/main.c}

Por otra parte, según la definición del código del CANBus (ver código \ref{lst:can_c}),
es necesario llamar a la función \texttt{CAN\_Handle\_IRQ} desde la rutina de 
tratamiento de interrupción para recibir el mensaje de forma efectiva. Por ello, es
necesario editar el fichero \texttt{stm32f4xx\_it.c} y añadir la llamada a la
función, como se muestra en el código \ref{lst:canirq}:

\lstinputlisting[style=C,caption={Rutina de tratamiento de interrupción del CANBus.},label={lst:canirq},linerange={37-40,207-220},firstnumber=37]{source-code/node2/Src/stm32f4xx_it.c}
