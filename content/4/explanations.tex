\begin{itemize}
  \item En los códigos \ref{lst:speedtask} y \ref{lst:lightstask}, el mapeo se realiza con valores
        de entrada $\left[0, 255\right]$ porque el ADC de la placa
        es de 8 bits, por lo que su resolución máxima es 255.
  \item En diversos códigos (como \ref{lst:distancetask} o \ref{lst:braketask}) se
        utilizan eventos para la sincronización de tareas entre sí. Los eventos
        aparecen en la documentación estándar de FreeRTOS y constituyen un mecanismo
        muy sencillo y eficiente que respeta el tiempo real para bloquear y desbloquear
        tareas sin necesidad de programar la lógica subyacente. Un evento, en esencia,
        se conforma de $1 \dots n$ procesos que esperan y, en principio, un único proceso
        $k$ que ``produce'' el evento. En ese instante, aquellas tareas que estaban
        esperando al evento se desbloquean y prosiguen con su ejecución; mientras tanto,
        el proceso $k$ reiniciaría el evento de forma que nuevas tareas pueden esperar
        a que se produzca.

        De esta manera, una tarea esporádica estaría esperando a que un evento se
        produzca y existiría una tarea periódica activadora la cual indicaría
        mediante dicho evento a la tarea esporádica que se tiene que ejecutar.
  \item En el código \ref{lst:wheel-task} se esperan 13 iteraciones que equivalen a un
        tiempo de $\numprint[s]{5.2}$ (en lugar de los $\numprint[s]{5}$ pedidos). Esto
        es debido a que el periodo no es múltiplo, por lo que se comete un error
        ``a la alta'' en lugar de ``a la baja''.
  \item Los ficheros \texttt{lock.h/.c} permiten trabajar con semáforos definidos de
        forma estática en lugar de forma dinámica (código \ref{lst:lock_c}). Sin embargo,
        para poder aprovechar dicha funcionalidad es necesario habilitar en el fichero
        \texttt{FreeRTOSConfig.h} la macro:
        \begin{center}
            \lstinline[style=C]!#define configSUPPORT_STATIC_ALLOCATION 1!
        \end{center}
        con un valor `1'. En otro caso, aunque se pase un parámetro válido, no se usará
        dicha funcionalidad.
  \item El pitido no se ha implementado a nivel de código ya que no venía especificado
        en los distintos diagramas a qué pin habría que conectarlo ni la lógica de
        control a usar para que funcionase con cierta intensidad.
  \item En el código \ref{lst:canbustasks} se han implementado dos tareas esporádicas
        para la gestión de los mensajes del CANBus porque dicho dispositivo recibe 
        mensajes desde una rutina de interrupción. Como se quieren
        actualizar variables las cuales utilizan un \texttt{lock} internamente, se
        deriva su gestión a un par de tareas esporádicas cuya única finalidad es la
        de actualizar el valor de los objetos protegidos.
  \item En los códigos \ref{lst:can_h} y \ref{lst:can_c}, como son en apariencia iguales
        para ambos nodos, es necesario definir una macro para que la versión del CANBus
        se adapte a la placa en que se ejecuta. De esta forma, en el nodo 2 habría que
        definir en alguna cabecera (se sugiere \texttt{FreeRTOSConfig.h}) la macro:
        \begin{center}
              \lstinline[style=C]!#define NODE_2!
        \end{center}
  \item En el código \ref{lst:riskstask} no se ha implementado la parte de interacción
        con el freno ya que la tarea \texttt{Cálculo distancia} (código \ref{lst:distancetask})
        se encarga individualmente de accionar el freno según la distancia de seguridad
        y una intensidad predefinida. Además, en las diapositivas de la especificación
        no se menciona esta comunicación con el nodo 1, por lo que se ha asumido que
        el verdadero encargado de accionar el freno es el nodo 1.
\end{itemize}